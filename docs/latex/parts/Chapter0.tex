\section{Введение}
\label{sec:Chapter0} \index{Chapter0}

Компьютерная графика с самого своего создания использовала для описания расположения объектов в пространстве набор \textit{(x, y, z, w)} координат, называемых \textit{однородными}. Их удобно использовать при описании евклидова пространства, поскольку первые три компоненты представляют из себя декартовы координаты. Эти координаты легко преобразовывать матрицами преобразования, такими как масштабирования, переноса и поворота.

Неевклидово пространство же не всегда можно описать такими однородными координатами. Более того, возможны пространства в которых нельзя ввести систему координат в принципе. Таким пространством, например, является пространство общей теории относительности, в котором происходит искривление.

Научное сообщество стремиться визуализировать черные дыры, червоточины и прочие искривления пространств. Видеоигровая индустрия тоже использует неевклидовы пространства, ярким примером является серии игр \textit{Portal}. Вышеперечисленное означает, что тема является актуальной в сообществе компьютерной графики.

Хорошим примером неевклидова пространства является черная дыра, потому что:
\begin{itemize}
  \item Актуально для научного сообщества.
  \item Можно реализовать общий метод рендеринга (визуализации) неевклидовых \linebreak
  пространств.
  \item Возможны оптимизации.
\end{itemize}

Именно поэтому в качестве рабочей сцены будет выбрана черная дыра. Таким образом, цели и задачи можно поставить следующие:

\textbf{\textit{Цель}}: Исследовать методы рендеринга неевклидовых пространств в реальном времени на примере искривленного черной дырой пространства.

\newpage

\textbf{\textit{Задачи}}:
\begin{enumerate}
  \item Реализовать общий случай рендеринга неевклидового пространства.
  \item Исследовать возможность оптимизаций.
  \item Сравнить производительность методов рендеринга.
\end{enumerate}

Для реализации поставленных задач будет использован графический интерфейс программирования приложений (API) \textit{Vulkan}, поскольку он является кроссплатформенным графическим API нового поколения (к новым относят \textit{DirectX 12}, \textit{Metal} и \textit{Vulkan}, к старым - \textit{OpenGL} и \textit{DirectX 11} и ниже), предоставляет более обширное управление для работы с видеокартой и обладает возможностью использования новых технологий последних лет (например сеточный шейдер (mesh shader) и трассировка лучей (ray tracing)).

\newpage
