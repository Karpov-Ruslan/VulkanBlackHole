\section{Введение}
\label{sec:Chapter0} \index{Chapter0}

Компьютерная графика с самого своего создания использовала для описания расположения объектов в пространстве набор \textit{(x, y, z, w)} координат, называемых \textit{однородными}. Их удобно использовать при описании евклидова пространства, поскольку первые три компоненты представляют из себя декартовы координаты. Эти координаты легко преобразовывать матрицами преобразования, такими как матрицы масштабирования, переноса и поворота.

Неевклидово пространство же не всегда можно описать такими однородными координатами. Более того, возможны пространства в которых нельзя ввести систему координат в принципе. Таким пространством, например, является пространство общей теории относительности, в котором происходит искривление.

Актуальность работы обусловлена потребностью визуализации черных дыр, червоточин и прочих искривленных пространств при реализации научных идей, в частности касающихся общей теории относительности. Не менее востребована необходимость визуализации неевклидовых пространств в видеоигровой индустрии, что подтверждается серией игр \textit{Portal}.

Выбор черной дыры в качестве неевклидова пространства в данной работе обусловлен следующими факторами:
\begin{itemize}
  \item Актуальностью применения исследуемых методов визуализации в научных целях.
  \item Возможностью реализации метода рендеринга (визуализации) неевклидовых пространств со сторонней геометрией.
  \item Возможностью реализации альтернативного метода рендеринга, оптимизирующего производительность.
\end{itemize}

Таким образом, в качестве рабочей сцены, на основании которой будут реализовываться различные методы рендеринга, выбрана черная дыра.

\textbf{\textit{Целью}} настоящей работы является исследование и реализация методов рендеринга неевклидовых пространств в режиме реального времени на примере искривленного черной дырой пространства. При этом при определенной доработке предлагаемых методов прогнозируется возможность их использования для более сложных случаев, таких как двойная статическая черная дыра или даже червоточина.

Для достижения указанной цели поставлены следующие \textbf{\textit{задачи}}:

\begin{enumerate}
  \item Реализовать алгоритм рендеринга черной дыры.
  \item Реализовать альтернативный алгоритм рендеринга черной дыры, оптимизирующий производительность.
  \item Реализовать  усовершенствованный алгоритм рендеринга черной дыры, позволяющий отрисовывать стороннюю геометрию.
  \item Сравнить производительность реализованных методов рендеринга.
\end{enumerate}

\newpage
