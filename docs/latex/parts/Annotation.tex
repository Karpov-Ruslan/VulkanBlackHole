\renewcommand{\abstractnamefont}{\normalfont\Large\bfseries}
\renewcommand{\abstracttextfont}{\normalfont\Huge}

\begin{center}
    \textbf{Аннотация}
\end{center}

    В данной работе исследуются методы визуализации неевклидовых пространств в режиме реального времени с точки зрения производительности и качества получаемых изображений.
	Рассмотрены такие методы как марширование и трассировка лучей, а также метод предварительного рассчета.
	Как основной результат работы, представлен приемлемый способ рендеринга сложных сцен в пространстве-времени с сильным искривлением.
	В качестве примера метрики пространства времени выбрана метрика гравитационного поля \emph{черной дыры}, для которой уже получены референсные изображения: \cite{SHolloway_2020}, \cite{yukterez} и др.
	Предложенный способ может быть легко расширен для более сложных случаев, таких как метрика поля двойной статической черной дыры \cite{Yurtsever_1995} или даже Червоточин \cite{wormhole}.
	Разработка методов ведется с использованием графического API Vulkan\cite{vulkan}.
    \vfill

    \vfill

\newpage