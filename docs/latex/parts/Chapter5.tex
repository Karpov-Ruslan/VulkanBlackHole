\section{Выводы}
\label{sec:Chapter5} \index{Chapter5}

Проведенное реализация и последующее сравнение методов визуализации неевклидовых пространств позволило определить их преимущества и недостатки.

Использование итеративного подхода к решению дифференциальных уравнений в качестве математического аппарата позволило реализовать различные методы визуализации черной дыры, включая оптимизированный метод предрасчитанных данных, а также метод визуализации черной дыры со сторонней геометрией посредством марширования и трассировки лучей.

Проведено сравнение производительности реализованных методов, показывающий огромную эффективность использования оптимизированных методов, а также малую производительность усовершенствованного трассировкой лучей метода визуализации черной дыры со сторонней геометрией по сравнению с другими реализованными методами.

При определенных доработках описанные в работе алгоритмы рендеринга черной дыры можно использовать для визуализации более сложных случаев искривленного пространства, таких, как двойная статическая черная дыра или даже червоточина.

Реализация исследованных методов рендеринга черной дыры представлена в \cite{github}

\newpage