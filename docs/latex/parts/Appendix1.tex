\appendix
\section*{Приложение A: Визуализация двойной черной дыры}
\addcontentsline{toc}{section}{Приложение A: Визуализация двойной черной дыры}
\label{sec:Apendix1} \index{Apendix1}

Рассмотрим более сложный случай, такой как двойная черная дыра.  К сожалению, не существует аналитического вида метрики для общего случая 
вращающихся друг относительно друга черных дыр, поскольку при этом возникают гравитационные волны, которые, в свою очередь, сами являются истоничком  неевклидовости пространства. Однако, есть выражения для  \emph{гипотетического} случая статических (экстремально)  заряженных черных дыр, заряд которых компенсирует взаимное притяжение. 

Такой статический случай описывается метрикой Райсснера-Нордстрёма \cite{Contopoulos2004ChaosIT}
для нескольких черных дыр:
\begin{equation}
\label{eq:reissner-nordstrom} 
ds^2 = \frac{1}{U^2}dt^2 - U^2(dx^2+dy^2+dz^2)
\end{equation}
где 
$$
	U = (1+\frac{m_1}{r_1}+\frac{m_2}{r_2}),
$$
$m_1$ и $m_2$ -- массы черных дыр, а  $r_1$, $r_2$ -- расстояния до черных дыр, которые располагаются вдоль оси $z$ в точках $z =\pm 1$, соответственно,
$r_1 = \sqrt{x^2+y^2+(z-1)^2}$ и $r_2 = \sqrt{x^2+y^2+(z+1)^2}$.

Для отображения сцен в данной метрике нам потребуется уравнение нуль-геодезической (поскольку это является траекторией света), выраженной через символы Кристоффеля $\Gamma^i_{jk}$:
\begin{equation}
\label{eq:geodesic}
	\frac{d^2 x_i}{d\lambda^2} = -\Gamma^i_{jk}\frac{d x_j}{d\lambda}\frac{d x_k}{d\lambda}
\end{equation}


Используя выражение для $\Gamma^i_{jk}$ и принимая во внимание, что метрика диагональна, получаем $\Gamma^i_{jk} = \frac{1}{2}g^{ii}(\partial_k g_{ij} + \partial_j g_{ik}-\partial_i g_{jk})$.
Если рассмотреть производные компонент метрического тензора, то можно наблюдать следющую симметрию:
\begin{equation}
	\begin{array}{cl}
		\partial_t g_{\cdot\cdot} = 0 &  \\
		\partial_a g_{bc} = 0, & a\ne b\ne c\ne a \\
		\partial_c g_{kk} = 2\Phi_c U, & k,c = \{x, y, z\}  \\
		\partial_c g_{tt} = 2\Phi_c/U^3, & c = \{x, y, z\} 
	\end{array}
\end{equation}
Здесь введены вспомогательные функции $\Phi_c$, которые появляются в результате дифференцирования:
$$
\Phi_x = \Phi x,\; \Phi_y = \Phi y,\; \Phi_z = \Phi z - \xi_z,
$$

$$
	\Phi = \frac{m_1}{r_1^3} +\frac{m_2}{r_2^3},\;  \xi_z = \frac{m_1 z_1}{r_1^3} + \frac{m_2 z_2}{r_2^3} 
$$

Соответственно, ненулевые символы Кристоффеля выглядят следующим образом: 
$$
\Gamma^t = \left[\begin{matrix}
	0 & \Phi_x/U & \Phi_y/U & \Phi_z/U \\
	\Phi_x/U & 0 & 0 & 0 \\
	\Phi_y/U & 0 & 0 & 0 \\
	\Phi z/U & 0 & 0 & 0
\end{matrix}\right]
$$
Если $x$ -- пространственная координата и $\alpha\ne x$, то 
$$
	\Gamma^x_{\alpha x} = -\frac{1}{2U^2}\partial_\alpha g_{xx} = -\Phi_\alpha/U
$$
$$
	\Gamma^x_{xx} = -\frac{1}{2U^2}\partial_x g_{xx} = -\Phi_x/U
$$
$$ 
	\Gamma^x_{\alpha\alpha} = -\frac{1}{2U^2}(-\partial_x g_{\alpha\alpha}) = \Phi_x/U
$$

В результате, имеем:

$$
\Gamma^x = \left[\begin{matrix}
	\Phi_x/U^5 & 0         &     0     &    0 \\
	0          & -\Phi_x/U & -\Phi_y/U & -\Phi_z/U \\
	0          & -\Phi_y/U & \Phi_x/U &  0 \\
	0          & -\Phi_z/U &  0       &  \Phi_x/U 
\end{matrix}\right]
$$
аналогично, для остальных пространственных координат получаем:
$$
\Gamma^y = \left[\begin{matrix}
	\Phi_y/U^5 & 0         &     0     &    0 \\
	0          & \Phi_y/U & -\Phi_x/U & 0 \\
	0          & -\Phi_x/U & -\Phi_y/U &  -\Phi_z/U \\
	0          & 0 &   -\Phi_z/U     &  \Phi_y/U 
\end{matrix}\right]
$$
$$
\Gamma^z = \left[\begin{matrix}
	\Phi_z/U^5 & 0         &     0     &    0 \\
	0          & \Phi_z/U  & 0        & -\Phi_x/U \\
	0          & 0         & \Phi_z/U &  -\Phi_y/U \\
	0          & -\Phi_x/U &  -\Phi_y/U     &  -\Phi_z/U 
\end{matrix}\right]
$$

Таким образом получен окончательный вид выражения для \eqref{eq:geodesic}. Его можно заменить вместо \eqref{eq:diffur} в методе Рунге-Кутты и итерироваться с выбранным шагом аффиного параметра $\Delta\lambda$.

Аффинный параметр $\lambda$ - это значение, параметризующее\linebreak нуль-геодезическую линейно, то есть если линейно изменять аффинный параметр $\lambda$, то тогда нуль-геодезическая будет делиться на равные по длине части в пространстве-времени (важно, что не просто в пространстве). Значит нужно изменять шаг итерации $\Delta\lambda$ в зависимости от расстояний до центров черных дыр, поскольку в дали от черных дыр малый шаг не нужен в связи с малыми искривлениями пространства.

\begin{equation}
    \Delta\lambda \sim \min{(r_1/m_1, r_2/m_2)}^2
\end{equation}

Методы Рунге-Кутты с переменным шагом называются \textit{адаптивными}.

Немного модифицируя алгоритм из раздела \ref{subsec:new_algos} вышеописанными правками, а также изменением краевого случая, касающегося попадания в горизонт событий, получаем следующее итоговое изображение:

\begin{figure}[h]
    \centering
    \includegraphics[width=1.0\linewidth]{twoBlackHoles}
    \caption{Система из двух черных дыр: красной и синей.}
    \label{fig:two_black_holes}
\end{figure}

\newpage