\section{Алгоритм рендеринга черной дыры}
\label{sec:Chapter1} \index{Chapter1}

Рендеринг реалистичного изображения в компьютерной графике можно сравнивать с фотосъемкой. И там, и там конечным результатом является изображение, которое получается на основании свойств света, попавшего в камеру. Пользуясь этим, можно упростить законы выбранного неевклидова пространства, используя их только для света (фотонов).

Целью данной главы является построение алгоритма рендеринга черной дыры, результатом которого является итоговый цвет.

\subsection{Метрика Шварцшильда}

В науке наиболее популярным методом описания пространства является метрика - симметричное тензорное поле ранга (0,2) на гладком многообразии, посредством которого задаётся скалярное произведение векторов в касательном пространстве. Иначе говоря, благодаря метрике можно понять как при малых изменениях локальных (т.е. применимых только в локальной области) координат $(dx, dy, ...)$ изменяется расстояние $ds$ между точками в пространстве. Примеры метрик:
\begin{itemize}
  \item Евклидова: $ds^2=dx^2+dy^2+dz^2$.
  \item Сферическая (двумерная): $ds^2=dx^2+cos^2(\frac{y}{R})dy^2$.
  \item Минковского: $ds^2=c^2dt^2-dx^2-dy^2-dz^2$.
\end{itemize}

Для описания выбранной рабочей сцены, т.е. для черной дыры (предполагая, что она не заряжена и не вращается), существует метрика Шварцшильда в сферических координатах $(t, r, \theta, \phi)$:
\begin{equation}
\label{eq:schwarzschild_metric}
    ds^2 = c^2d\tau^2 = \left(1-\frac{r_s}{r}\right)c^{2}dt^{2} - \left(1-\frac{r_s}{r}\right)^{-1}dr^2 - r^2\left(d\theta^2+sin^2(\theta)d\phi^2\right)
\end{equation}
, где $\tau$ - собственное время и $r_s = \frac{2GM}{c^2}$ - \textit{радиус Шварцшильда}, который определяет область вокруг ЧД массой $M$, которую луч света покинуть не сможет. Считаем, что центр черной дыры расположен в начале координат.

Для частиц в течение движения сохраняются энергия и угловой момент (константы движения):
\begin{equation}
\label{eq:motion_constant_E}
    E = \left(1-\frac{r_s}{r}\right)\frac{dt}{d\tau}.
\end{equation}

\begin{equation}
\label{eq:motion_constant_L}
    L = r^2\frac{d\phi}{d\tau}.
\end{equation}

Метрику можно преобразовать из \eqref{eq:schwarzschild_metric}, считая $c = 1$ и $\theta=\frac{\pi}{2}$, поскольку метрика Шварцшильда симметрична, а значит траектория луча всегда будет лежать в плоскости:
\begin{equation}
\label{eq:schwarzschild_metric1}
    c^2 = \left(1-\frac{r_s}{r}\right)c^2\left(\frac{dt}{d\tau}\right)^2 - \left(1-\frac{r_s}{r}\right)^{-1}\left(\frac{dr}{d\tau}\right)^2 - r^2\left(\frac{d\phi}{d\tau}\right)^2.
\end{equation}

Из \eqref{eq:motion_constant_E}, \eqref{eq:motion_constant_L} и \eqref{eq:schwarzschild_metric1} можно получить следующее выражение:
\begin{equation}
\label{eq:schwarzschild_metric2}
    \left(\frac{dr}{d\tau}\right)^2 = \frac{E^2}{m^2c^2} - \left(1-\frac{r_s}{r}\right)\left(c^2+\frac{L^2}{r^2}\right).
\end{equation}

Поскольку справедливо следующее равенство:
\begin{equation}
\label{eq:equation_of_diff}
    \left(\frac{dr}{d\phi}\right)^2 = \left(\frac{dr}{d\tau}\right)^2\left(\frac{d\tau}{d\phi}\right)^2 = \left(\frac{dr}{d\tau}\right)^2\left(\frac{r^2}{L}\right)^2.
\end{equation}

То используя \eqref{eq:schwarzschild_metric2} и \eqref{eq:equation_of_diff} можно получить уравнение:
\begin{equation}
\label{eq:schwarzschild_metric3}
    \left(\frac{dr}{d\phi}\right)^2 = \frac{r^4}{b^2} - \left(1-\frac{r_s}{r}\right)\left(\frac{r^4}{a^2} + r^2\right).
\end{equation}
, где $a = \frac{L}{c}$ и $b = \frac{mcL}{E}$ - константы.

Применяя замену переменной $u = \frac{1}{r}$, а также учитывая, что при стремлении массы частицы $m$ к нулю, $m \longrightarrow 0$, константа движения $a$ стремиться к бесконечности, $a \longrightarrow \infty$, упрощаем \eqref{eq:schwarzschild_metric3}:
\begin{equation}
\label{eq:schwarzschild_metric4}
    \left(\frac{du}{d\phi}\right)^2 = r_su^3 - u^2 + \frac{1}{b^2}.
\end{equation}

Дифференцируя \eqref{eq:schwarzschild_metric4}:
\begin{equation}
\label{eq:schwarzschild_metric5}
    2\frac{du}{d\phi}\frac{d^2u}{d\phi^2} = 3r_su^2\frac{du}{d\phi} - 2u\frac{du}{d\phi}.
\end{equation}

Итоговый результат из \eqref{eq:schwarzschild_metric5} выглядит следующим образом:
\begin{equation}
\label{eq:diffur}
    \frac{d^2u}{d\phi^2} = \frac{3}{2}r_su^2 - u.
\end{equation}

\subsection{Методы Рунге-Кутты}
\label{subsec:runge-kutte}

Дифференциальное уравнение \eqref{eq:diffur} не имеет явного решения, поэтому разумно использовать для его решения итеративный подход. Таким образом траектория (луч) света будут приближены итеративными точками (см. рис. \ref{fig:runge-kutte}).

\begin{figure}[h]
    \centering
    \includegraphics[width=0.8\linewidth]{BlackHole-RayMarching-Overview}
    \caption{Черной сплошной линией показана теоретическая траектория луча света в окрестности черной дыры, а красными точками - итерации метода Рунге-Кутты.}
    \label{fig:runge-kutte}
\end{figure}

Одними из итеративных подходов являются методы Рунге-Кутты, позволяющие итеративно находить решение дифференциальных уравнений с заданными условиями на сетке $\phi$ значений, обычно с постоянным шагом $\Delta \phi$:

\begin{equation}
\label{eq:runge-kutte}
    \frac{\mathbf{dy}}{d\phi} = \mathbf{f}(\phi, \mathbf{y}),
    \quad
    \left.\mathbf{y}\right|_{\phi=\phi_0}=\mathbf{y_0}.
\end{equation}

В случае уравнения \eqref{eq:diffur}:

\begin{equation*}
\label{eq:runge-kutte_y}
    \mathbf{y} =
    \begin{cases}
        u
        \\
        \frac{du}{d\phi}
    \end{cases}
\end{equation*}

\begin{equation*}
\label{eq:runge-kutte_f}
    \mathbf{f}(\phi, \mathbf{y}), =
    \begin{cases}
        \frac{du}{d\phi}
        \\
        \frac{3}{2}r_su^2 - u
    \end{cases}
\end{equation*}

Для данной работы были выбраны явные методы Рунге-Кутты порядков 1, 2 и 4. Разные порядки имеют разную точность, сведения представлены в таблице \ref{tab:runge-kutte}:

\begin{center}
    \begin{table}[h!]
        \centering
        \begin{tabular}{|c|c|}
            \hline
            Порядок метода Рунге-Кутты & Погрешность шага $\Delta h$ \\
            \hline
            Рунге-Кутты 1-го порядка (RK1) & $O(h^2)$ \\
            \hline
            Рунге-Кутты 2-го порядка (RK2) & $O(h^3)$ \\
            \hline
            Рунге-Кутты 4-го порядка (RK4) & $O(h^5)$ \\
            \hline
        \end{tabular}
        \caption{Сравнение погрешностей методов Рунге-Кутты.}
        \label{tab:runge-kutte}
    \end{table}
\end{center}


\subsection{Инициализация лучей света}
\label{subsec:ray_init}

Осталось только определить $y_0$ и $\phi_0$ из \eqref{eq:runge-kutte}. Начальные условия $y_0$ будут определяться на основе позиции и направления камеры (т.е. наблюдателя), а также угла обзора (field of view).

\begin{figure}[h]
    \centering
    \includegraphics[width=0.5\linewidth]{CameraRaysProblem}
    \caption{Малая часть испущенного света попадает в камеру.}
    \label{fig:camera_rays_problem}
\end{figure}

Стоит отметить, что, как было сказано выше, рендеринг - это свет попавший в камеру. Но испускать свет в сцене от всех источников освещения, а затем учитывать только тот свет, что попал в камеру - невыгодно, поскольку лишь малая часть действительно попадет в камеру (см. рис. \ref{fig:camera_rays_problem}). Поэтому в компьютерной графике давно используют закон обратимости света: гораздо выгоднее выпускать лучи из каждого пикселя камеры и затем просчитывать какой цвет они приобретут в ходе движения по сцене.

\begin{figure}[h]
    \centering
    \includegraphics[width=1.0\linewidth]{PinholeCameraModel}
    \caption{Модель камеры-обскуры (pinhole camera model).}
    \label{fig:pinhole_camera}
\end{figure}

В качестве модели камеры выбрана модель камеры-обскуры \linebreak(pinhole camera model) \cite[стр.~47]{marrs2021ray}, поскольку это одна из стандартных моделей камеры. Для ее настройки необходимо знать параметры камеры и угол обзора (поле зрения). Главным следствием использования этой модели является эквидистантность точек пересечения лучей с плоскостью изображения, лежащих на одной линии (см. рис. \ref{fig:pinhole_camera}).

Пусть изображение представляет из себя тензор второго ранга размером $(N, M)$. Индекс пикселя это упорядоченный набор двух неотрицательных чисел $(n, m)$, начиная с $0$ по каждому пикселю. Направление и позиция в декартовых координатах, а также угол обзора камеры обозначаются $\mathbf{d_c}$, $\mathbf{a_c}$ и $fov$ соответственно. В терминологии данной работы углом обзора будет называться угол между крайними лучами на одной горизонтальной прямой. Зная все это можно получить направления лучей каждого пикселя в декартовых координатах $\mathbf{d_0}$:

\begin{align*}
    x &= (n + 0.5)/N, \\
    y &= (m + 0.5)/M, \\
    scale_{horizontal} = scale_h &= \tan{\left(\frac{fov}{2}\right)}, \\
    scale_{vertical} = scale_v &= \left(\frac{N}{M}\right) \cdot scale_h, \\
    \mathbf{offset}_{horizontal} = \mathbf{offset}_h &= scale_h \cdot \left(\frac{\mathbf{d_c} \times \mathbf{e_z}}{\left|\mathbf{d_c} \times \mathbf{e_z}\right|}\right), \\
    \mathbf{offset}_{vertical} = \mathbf{offset}_v &= scale_v \cdot \left(\frac{\mathbf{d_c} \times \mathbf{offset_h}}{\left|\mathbf{d_c} \times \mathbf{offset_h}\right|}\right).
\end{align*}

Таким образом, направление луча пикселя в декартовых кооринатах $\mathbf{d_0}$ можно вычислить следующим образом:

\begin{equation}
\label{eq:init_rays}
    \mathbf{d_{nm}} = \mathbf{d_c} + \mathbf{offset}_{h} \cdot x + \mathbf{offset}_{v} \cdot y.
\end{equation}

\subsection{Переход из декартовых в полярные координаты плоскости вращения}
\label{subsec:transition_from_decart_to_polar}

Полученные начальные условия $\mathbf{d_0}$ из \eqref{eq:init_rays} получены в декартовых координатах, а метод Рунге-Кутты, изложенный в разделе \ref{subsec:runge-kutte}, описан в полярных координатах. Поэтому перед итеративным решением необходимо определить $y_0$ и $\phi_0$ из \eqref{eq:runge-kutte} на основании данных раздела \ref{subsec:ray_init}.

\newpage

\begin{figure}[h]
    \centering
    \includegraphics[width=1.0\linewidth]{rotation_plane}
    \caption{Разные плоскости вращения с точки зрения камеры.}
    \label{fig:rotation_plane}
\end{figure}

Для простоты примем $\phi_0 = 0$. Начальный радиус $r_0 = |\mathbf{a_c}|$, т.е. начальный обратный радиус $u_0 = \frac{1}{r_0} = \frac{1}{|\mathbf{a_c}|}$. Таким образом первая компонента $\mathbf{y_0}$ определена.

Для нахождения второй компоненты сперва нужно определить вектор вращения луча $\mathbf{w_{nm}}$ т.е. вектор, задающий плоскость вращения. В общем случае для каждого пикселя плоскость вращения будет отличаться (см. рис. \ref{fig:rotation_plane}).

\begin{equation}
\label{eq:w_ray}
    \mathbf{w_{nm}} = \mathbf{a_c} \times \mathbf{d_{nm}}.
\end{equation}

\newpage

\begin{figure}[h]
    \centering
    \includegraphics[width=0.8\linewidth]{BlackHole-RayAngle}
    \caption{Начальные условия (зеленым изображена траектория света, черным пунктиром - окружность, центр которой совпадает с центром черной дыры).}
    \label{fig:blackhole_rayAngle}
\end{figure}

Поскольку вторая компонента $\mathbf{y_0}$ это $\frac{du}{d\phi}$, то необходимо вычислить начальное значение этой производной (см. рис. \ref{fig:blackhole_rayAngle}), пользуясь методом малых перемещений:

\begin{equation}
\label{eq:dudr_init}
    \left.\frac{du}{d\phi}\right|_{\phi=\phi_0=0} = \frac{d\left(\frac{1}{r}\right)}{d\phi} = -\frac{1}{r^2}\left(\frac{dr}{d\phi}\right) = -\frac{\tan{\alpha}}{r} = -\frac{1}{r}\left(\frac{\mathbf{a_c} \cdot \mathbf{d_{nm}}}{\left|\mathbf{w_{nm}}\right|}\right).
\end{equation}

Подытоживая, начальное условие $\mathbf{y_0}$ будет равно:

\begin{equation}
\label{eq:runge-kutte_y0}
    \mathbf{y_0} =
    \begin{cases}
        \frac{1}{\left|\mathbf{a_c}\right|}
        \\
        -\frac{1}{r}\left(\frac{\mathbf{a_c} \cdot \mathbf{d_{nm}}}{\left|\mathbf{w_{nm}}\right|}\right)
    \end{cases}
\end{equation}

\newpage

\subsection{Краевые случаи}
\label{subsec:corner_cases}
Для черной дыры выделяются два краевых случая:

\begin{enumerate}
    \item Попадание луча в горизонт событий.
    \item Уход луча на бесконечность.
\end{enumerate}

Рассмотрим каждые из них отдельно.

\subsubsection{Попадание луча в горизонт событий}
\label{subsubsec:events_horizon}

Луч, попавший в горизонт событий, уже никогда не выйдет из него, поэтому можно сказать, пользуясь законом обратимости света, что такие лучи приобретают черный цвет (если по пути до этого не пересеклись с другим объектом, разумеется).

Поскольку \textit{радиус Шварцшильда} $r_s$ является радиусом горизонта событий, то на каждой итерации решения уравнения \eqref{eq:diffur} методом Рунге-Кутты необходимо проводить проверку на нахождение внутри шара горизонта событий. То есть если на $i$-ой итерации, $r_i = \frac{1}{u_i} < r_s$, то нужно остановить итерирование и возвращать в качестве результата алгоритма рендеринга черной дыры полученный на предыдущих итерациях итоговый цвет.

\subsubsection{Уход луча на бесконечность}
\label{subsubsec:goes_to_infinity}

Луч, находящийся далеко от черной дыры, движется по почти прямой траектории. В таком случае можно считать, что направление луча не измениться, а значит можно сразу окрасить луч света в какой-то цвет на основании \textit{дальнего} окружения черной дыры.

\newpage

\begin{figure}[h]
    \centering
    \includegraphics[width=1.0\linewidth]{cubemap}
    \caption{Кубическая текстура звездного неба.}
    \label{fig:cubemap}
\end{figure}

Черные дыры - это массивные объекты, находящиеся в космосе. Поэтому в качестве окружения разумно выбрать звездное небо (см. рис. \ref{fig:cubemap}) и в случае ухода луча на бесконечность брать значения, соответствующие направлению луча вдали от черной дыры. из этого звездного неба.

Осталось определить условие ухода луча на бесконечность. В общем случае на каждой $i$-ой итерации стоило бы проверять, как далеко свет ушел, но для черной дыры можно сделать упрощение, которое несет в себе еще и решение различных визуальных проблем: когда луч уходит все дальше и дальше от черной дыры, то $\phi$ из \eqref{eq:diffur} почти перестает меняться, что может вызывать проблемы с постоянным шагом $\Delta\phi$ в методе Рунге-Кутты. Упрощенное условие ухода на бесконечность выглядит следующим образом (см. рис. \ref{fig:blackhole_rayAngle}):

\begin{equation}
\label{eq:goes_to_infinity}
    \tan{\alpha} > tan_{crit} \quad \textit{или} \quad \left(\frac{du}{d\phi}\right)_i < -tan_{crit} \cdot u.
\end{equation}
, где $tan_{crit}$ - критическое значение тангенса угла $\alpha$ к окружности с центром, совпадающим с центром черной дыры.

\subsection{Аккреционный диск}
\label{subsec:accr_disk}

Все прошлые разделы описывали реализацию \textit{гравитационного линзирования}. Однако стоит отметить, что по соседству с черной дырой часто оказывается вращающийся горячий газовый диск, именуемый \textit{аккреционным диском}. Аккреционный диск по своим размерам сопоставим с радиусом Шварцшильда во всех направлениях (в том числе и в толщине) и поскольку это горячий газ, то его можно воспринимать как полупрозрачный объект.

Сложность визуализации аккреционного диска зависит от желаемого уровня реалистичности. Для данной работы достаточно ограничиться следующими параметрами аккреционного диска: вектор вращения диска $\mathbf{w_{accr}}$, внутренним $r_{min}$ и внешним $r_{max}$ радиусами, толщиной диска $H_{accr}$, а также цветом диска зависящим только от радиуса и расстоянием $h_{accr}$ от плоскости диска, задаваемой вектором вращения $\mathbf{w_{accr}}$. Этих параметров хватит для описания радиальной плотности $\rho_{accr}$ аккреционного диска:

\begin{equation}
\label{eq:accr_disk}
    \rho_{accr} = \max{\left(\text{noise}(r_{accr}) - \frac{\left|h_{accr}\right|)}{H_{accr}}, 0\right)},
    \quad
    r_{min} \le r_{accr} \le r_{max}.
\end{equation}
, где $r_{accr}$ - радиус в полярных координатах плоскости вращения аккреционного диска, а $\text{noise}(x)$ - функция плавной случайности, значения которой являются псевдослучайными, но при этом плавно меняются вдоль оси $x$.

Формулы для вычисления параметров $h_{accr}$ и $r_{accr}$ при известных декартовых координатах $\mathbf{a}$:

\begin{equation}
\label{eq:h_accr}
    h_{accr} = -\left(\mathbf{a} \cdot \mathbf{w_{accr}}\right)
\end{equation}

\begin{equation}
\label{eq:r_accr}
    r_{accr} = \left|\mathbf{a} + h_{accr}\mathbf{w_{accr}}\right|
\end{equation}

Плотность $\rho_{accr}$ используется на каждом шаге итерации, чтобы добавить плотность в данной точку итерации, помноженную на цвет аккреционного диска, к итоговому цвету.

\subsection{Переход из полярных координат плоскости вращения в декартовы}
\label{subsec:transition_from_polar_to_decart}

Для отрисовки аккреционного диска необходимо перейти из полярных координат $\left(\frac{1}{u}, \phi\right)$, в которых решается дифференциальное уравнение \eqref{eq:diffur} в декартовы $\left(\mathbf{a}, \mathbf{d}\right)$, где $\mathbf{a}$ - координаты луча, а $\mathbf{d}$ - направление луча. Это необходимо сделать, поскольку в декартовых координатах проще вычислить $r_{accr}$ и $h_{accr}$. Разумно разбить обсуждаемый переход на два последовательных этапа: поворот и масштабирование.

Поворот координат определяется начальным вектором, углом поворота и поскольку используются трехмерные декартовы координаты, то еще необходимо использовать вектор плоскости. Начальным вектором в данном случае будет нормализованный вектор позиции камеры $\mathbf{a_c}$ из раздела \ref{subsec:ray_init}, углом поворота будет угол $\phi$ из уравнения \eqref{eq:diffur}, а вектором плоскости - вектор вращения $\mathbf{w_{nm}}$ из все того же раздела \ref{subsec:ray_init}. Таким образом для искомой позиции $\mathbf{a}$ можно найти нормализованный вектор $\frac{\mathbf{a}}{\left|\mathbf{a}\right|}$ можно найти по следующей формуле:

\begin{equation}
\label{eq:rotate_pos}
    \frac{\mathbf{a}}{\left|\mathbf{a}\right|} = \left( \frac{\mathbf{a_{с}}}{\left|\mathbf{a_{с}}\right|} \right) \cdot \cos{\phi} + \left(\frac{\mathbf{w_{nm}}}{\left|\mathbf{w_{nm}}\right|} \times \frac{\mathbf{a_c}}{\left|\mathbf{a_c}\right|} \right) \cdot \sin{\phi}.
\end{equation}

Для нахождения вектора $\mathbf{a}$ нужно произвести масшитабирование, т.е. домножить найденное в \eqref{eq:rotate_pos} на $\left|\mathbf{a}\right| = \frac{1}{u} = r$ - радиус.

\begin{figure}[h]
    \centering
    \includegraphics[width=0.7\linewidth]{TransformDirection}
    \caption{Перпендикуляр к движению луча (угловой вектор) и радиальный вектор направления $\mathbf{d}$.}
    \label{fig:transform_direction}
\end{figure}

Помимо найденной позиции $\mathbf{a}$ необходимо также найти направление $\mathbf{d}$ луча в точке $\mathbf{a}$:

\begin{equation}
\label{eq:rotate_dir}
    \mathbf{d} = \frac{\left(\frac{\mathbf{w_{nm}}}{\left|\mathbf{w_{nm}}\right|} \times \frac{\mathbf{a}}{\left|\mathbf{a}\right|}\right) - \mathbf{a}\left(\frac{du}{d\phi}\right)}{\left| \left(\frac{\mathbf{w_{nm}}}{\left|\mathbf{w_{nm}}\right|} \times \frac{\mathbf{a}}{\left|\mathbf{a}\right|}\right) - \mathbf{a}\left(\frac{du}{d\phi}\right) \right|}
\end{equation}
, где $\left(\frac{\mathbf{w_{nm}}}{\left|\mathbf{w_{nm}}\right|} \times \frac{\mathbf{a}}{\left|\mathbf{a}\right|}\right)$ - перпендикуляр к позиции $\mathbf{a}$ в плоскости вращения, направленный вдоль движения (угловой вектор направления $\mathbf{d}$), $\mathbf{a}\left(\frac{du}{d\phi}\right) = \frac{\mathbf{a}}{\left|\mathbf{a}\right|}\left(\frac{du}{d\phi} \cdot r\right) = \frac{\mathbf{a}}{\left|\mathbf{a}\right|} \tan{\alpha}$ - радиальный вектор направления $\mathbf{d}$ (см. рис. \ref{fig:transform_direction}). 

\subsection{Обзор итогового алгоритма}
\label{subsec:algos}

Итоговый алгоритм выглядит следующим образом:

\begin{enumerate}
\label{enum:algos}
    \item Инициализация лучей света (раздел \ref{subsec:ray_init}).
    \item Переход из декартовых координат в полярные координаты плоскости вращения (раздел \ref{subsec:transition_from_decart_to_polar}).
    \item В бесконечном цикле решения дифференциального уравнения методом Рунге-Кутты:
    \label{item:infinity_loop}
    \begin{enumerate}
        \item Проверка краевых случаев (раздел \ref{subsec:corner_cases}):
        \begin{enumerate}
            \item Проверка на нахождение в горизонте событий (раздел \ref{subsubsec:events_horizon}).
            \item Проверка на уход в бесконечность (раздел \ref{subsubsec:goes_to_infinity}).
            \label{item:goes_to_infinity}
        \end{enumerate}
        \item Получение новой точки итерации методом Рунге-Кутты (раздел \ref{subsec:runge-kutte}).
        \item Переход из полярных координат плоскости вращение в декартовы (раздел \ref{subsec:transition_from_polar_to_decart}).
        \item Добавление к итоговому цвету цвета аккреционного диска (раздел \ref{subsec:accr_disk}).
    \end{enumerate}
\end{enumerate}

Полученный алгоритм не следует реализовывать буквально в вышеописанном виде. Бесконечный цикл в пункте \ref{item:infinity_loop} желательно сделать конечным, но с большой верхней границей количества итераций, поскольку лучи теоретически могут попасть в фотонную сферу - область вокруг черной дыры, в которой лучи двигаются по окружности бесконечно и таким образом не достигая краевых случаев. По этой причине окончание конечного цикла стоит обработать отдельно. В данной работе обработка такая же, как и в случае с уходом в бесконечность в пункте \ref{item:goes_to_infinity}.

Итоговый алгоритм является примером \textit{марширования лучей} \linebreak\textit{(ray marching)}.

\begin{figure}[h]
    \centering
    \includegraphics[width=1.0\linewidth]{BlackHoleRayMarching}
    \caption{Рендеринг черной дыры с использованием марширования лучей.}
    \label{fig:black_hole_ray_marching}
\end{figure}

\newpage
